%
% this file is encoded in utf-8
% v1.7
% do not change the content of this file
% unless the thesis layout rule is changed
% 無須修改本檔內容,除非校方修改了
% 封面、書名頁、中文摘要、英文摘要、誌謝、目錄、表目錄、圖目錄、符號說明
% 等頁之格式
% this file is encoded in utf-8
%v1.7

% make the line spacing in effect
\renewcommand{\baselinestretch}{\mybaselinestretch}
\large % it needs a font size changing command to be effective

% default variables definitions
% 注意!!此處只是預設值,不需更改此處
% 注意!!此處只是預設值,不需更改此處
% 注意!!此處只是預設值,不需更改此處
% 注意!!此處只是預設值,不需更改此處
% 注意!!此處只是預設值,不需更改此處  很重要講五次
% 請更改 my_names.tex 內容
\newcommand\cTitle{論文題目}
\newcommand\eTitle{MY THESIS TITLE}
\newcommand\myCname{王鐵雄}
\newcommand\myEname{Aron Wang}
\newcommand\myStudentID{M9315048}
\newcommand\advisorCnameA{南宮明博士}
\newcommand\advisorEnameA{Dr.~Ming Nangong}
\newcommand\advisorCnameB{李斯坦博士}
\newcommand\advisorEnameB{Dr.~Stein Lee}
\newcommand\advisorCnameC{徐 石博士}
\newcommand\advisorEnameC{Dr.~Sean~Hsu}
\newcommand\univCname{國立交通大學}
\newcommand\univEname{National Chaio Tung University}
\newcommand\deptCname{電信工程學系}
\newcommand\fulldeptEname{Graduate School of Communication Engineering}
\newcommand\deptEname{Communication Engineering}
\newcommand\collEname{College of Electrical Engineering and Computer Science}
\newcommand\degreeCname{碩士}
\newcommand\degreeEname{Master of Communication Engineering}
\newcommand\cYear{九十四}
\newcommand\cMonth{六}
\newcommand\cDay{十}
%\newcounter{eYear}
\newcommand\eYear{2006}
\newcommand\eMonth{June}
\newcommand\ePlace{Hsinchu, Taiwan}


 % user's names; to replace those default variable definitions
%
% this file is encoded in utf-8
% v1.7
% 填入你的論文題目、姓名等資料
% 如果題目內有必須以數學模式表示的符號,請用 \mbox{} 包住數學模式,如下範例
% 如果中文名字是單名,與姓氏之間建議以全形空白填入,如下範例
% 英文名字中的稱謂,如 Prof. 以及 Dr.,其句點之後請以不斷行空白~代替一般空白,如下範例
% 如果你的指導教授沒有如預設的三位這麼多,則請把相對應的多餘教授的中文、英文名
%    的定義以空的大括號表示
%    如,\renewcommand\advisorCnameB{}
%          \renewcommand\advisorEnameB{}
%          \renewcommand\advisorCnameC{}
%          \renewcommand\advisorEnameC{}

% 論文題目 (中文)
\renewcommand\cTitle{%我的碩士論文題目 
無線網路巴拉巴拉巴拉巴拉
}

% 論文題目 (英文)
\renewcommand\eTitle{%My Thesis Title  
Fire! Neo armstrong cyclone jet armstrong cyclone jet cannon!
%My Thesis Title  \mbox{$\cal{H}_\infty$} and \mbox{Al$_x$Ga$_{1-x}$As}
}

% 我的姓名 (中文)
\renewcommand\myCname{金城武}

% 我的姓名 (英文)
\renewcommand\myEname{I see You}

%我的學號
\renewcommand\myStudentID{7533967}

% 指導教授A的姓名 (中文)
\renewcommand\advisorCnameA{鐵拳無敵}

% 指導教授A的姓名 (英文)
\renewcommand\advisorEnameA{Dr.~Wu}

% 指導教授B的姓名 (中文)
\renewcommand\advisorCnameB{}

% 指導教授B的姓名 (英文)
\renewcommand\advisorEnameB{}

% 指導教授C的姓名 (中文)
\renewcommand\advisorCnameC{}

% 指導教授C的姓名 (英文)
\renewcommand\advisorEnameC{}

% 校名 (中文)
\renewcommand\univCname{國立交通大學}

% 校名 (英文)
\renewcommand\univEname{National Chiao Tung University}

% 系所名 (中文)
\renewcommand\deptCname{電信工程研究所}

% 系所全名 (英文)
\renewcommand\fulldeptEname{Institute of Communications Engineering }

% 系所短名 (英文, 用於書名頁學位名領域)
\renewcommand\deptEname{Communications Engineering}

% 學院英文名 (如無,則以空的大括號表示)
\renewcommand\collEname{College of Electrical and Computer Engineering}

% 學位名 (中文)
\renewcommand\degreeCname{碩士}
%\renewcommand\degreeCname{博士}

% 學位名 (英文)
\renewcommand\degreeEname{Master}
%\renewcommand\degreeEname{Doctor of Philosoph}

% 口試年份 (中文、民國)
\renewcommand\cYear{一零四}

% 口試月份 (中文)
\renewcommand\cMonth{七} 

% 口試月份 (中文)
\renewcommand\cDay{七} 

% 口試年份 (阿拉伯數字、西元)
\renewcommand\eYear{2015} 

% 口試月份 (英文)
\renewcommand\eMonth{July}

% 學校所在地 (英文)
\renewcommand\ePlace{Hsinchu, Taiwan}

%畢業級別;用於書背列印;若無此需要可忽略
\newcommand\GraduationClass{96}

%%%%%%%%%%%%%%%%%%%%%%
\newcommand\itsempty{}
%%%%%%%%%%%%%%%%%%%%%%%%%%%%%%%
%       nctu ccover1 封面
%%%%%%%%%%%%%%%%%%%%%%%%%%%%%%%
%
\begin{titlepage}
% no page number
% next page will be page 1

% aligned to the center of the page
\begin{center}
% font size (relative to 12 pt):
% \large (14pt) < \Large (18pt) < \LARGE (20pt) < \huge (24pt)< \Huge (24 pt)
%

%\vspace{1cm}
%\makebox[6cm][s]{\textbf{\Huge{\degreeCname 論文}}}\\ %顯示論文種類 (中文)
%\vspace{1cm}

\Huge{\univCname}\\
\Huge{\deptCname}\\
\Huge{\degreeCname 論文}\\
\vspace{2cm}



%
% Set the line spacing to single for the titles (to compress the lines)
\renewcommand{\baselinestretch}{1}   %行距 1 倍
%\large % it needs a font size changing command to be effective
\LARGE{\cTitle}\\  % 中文題目
%
\vspace{1cm}
%
\LARGE{\eTitle}\\ %英文題目
\vspace{5cm}
% \makebox is a text box with specified width;
% option s: stretch
% use \makebox to make sure
% 「研究生:」 與「指導教授:」occupy the same width
\hspace{4.5cm} \makebox[3cm][s]{\Large{研 究 生:}}
\Large{\myCname}  % 顯示作者中文名
\hfill \makebox[1cm][s]{}\\
%
%\vspace{0.3cm}
%\hspace{4.5cm} \makebox[3cm][s]{\Large{學號:}}
%\Large{\myStudentID}  %顯示指導教授A中文名
%\hfill \makebox[1cm][s]{}\\
%
\vspace{1cm}
\hspace{4.5cm} \makebox[3cm][s]{\Large{指導教授:}}
\Large{\advisorCnameA~教授}  %顯示指導教授A中文名
\hfill \makebox[1cm][s]{}\\
%
% 判斷是否有共同指導的教授 B
\ifx \advisorCnameB  \itsempty
\relax % 沒有 B 教授,所以不佔版面,不印任何空白
\else
% 共同指導的教授 B
\hspace{4.5cm} \makebox[3cm][s]{}
\Large{\advisorCnameB~教授}  %顯示指導教授B中文名
\hfill \makebox[1cm][s]{}\\
\fi
%
% 判斷是否有共同指導的教授 C
\ifx \advisorCnameC  \itsempty
\relax % 沒有 C 教授,所以不佔版面,不印任何空白
\else
% 共同指導的教授 C
\hspace{4.5cm} \makebox[3cm][s]{}
\Large{\advisorCnameC~教授}  %顯示指導教授C中文名
\hfill \makebox[1cm][s]{}\\
\fi
%
\vfill
\makebox[10cm][s]{\Large{中華民國\cYear 年\cMonth 月}}%
%
\end{center}
% Resume the line spacing to the desired setting
\renewcommand{\baselinestretch}{\mybaselinestretch}   %恢復原設定
% it needs a font size changing command to be effective
% restore the font size to normal
\normalsize
\end{titlepage}


%%%%%%%%%%%%%%%%%%%%%%%%%%%%%%%
%       nctu cover2 封面
%%%%%%%%%%%%%%%%%%%%%%%%%%%%%%%
%
\begin{titlepage}
% no page number
% next page will be page 1

% aligned to the center of the page
\begin{center}
% font size (relative to 12 pt):
% \large (14pt) < \Large (18pt) < \LARGE (20pt) < \huge (24pt)< \Huge (24 pt)
%

\Large{\eTitle}\\
\Large{\cTitle}\\
\vspace{1cm}

%
% Set the line spacing to single for the titles (to compress the lines)
\renewcommand{\baselinestretch}{1}   %行距 1 倍
%\large % it needs a font size changing command to be effective
% \makebox is a text box with specified width;
% option s: stretch
% use \makebox to make sure
% 「研究生:」 與「指導教授:」occupy the same width
\begin{flushleft}
\makebox[3cm][s]{\Large{研究生:}}
\makebox[3cm][l]{\Large{\myCname}}  % 顯示作者中文名
\hspace{1cm}
\makebox[3cm][s]{\Large{Student:}}
\makebox[4cm][l]{\Large{\myEname}}
\hfill
%
%\vspace{0.3cm}
%\hspace{4.5cm} \makebox[3cm][s]{\Large{學號:}}
%\Large{\myStudentID}  %顯示指導教授A中文名
%\hfill \makebox[1cm][s]{}\\
%
\vspace{0.5cm}
\makebox[3cm][s]{\Large{指導教授:}}
\makebox[3cm][l]{\Large{\advisorCnameA}}  %顯示指導教授A中文名
\hspace{1cm}
\makebox[3cm][s]{\Large{Advisor:}}
\makebox[4cm][l]{\Large{\advisorEnameA}}
\hfill
%
% 判斷是否有共同指導的教授 B
\ifx \advisorCnameB  \itsempty
\relax % 沒有 B 教授,所以不佔版面,不印任何空白
\else
% 共同指導的教授 B
\makebox[3cm][s]{}
\makebox[3cm][l]{\Large{\advisorCnameB}}  %顯示指導教授B中文名
\hspace{1cm}
\makebox[3cm][s]{}
\makebox[4cm][l]{\Large{\advisorEnameB}}
\hfill
\fi
%
% 判斷是否有共同指導的教授 C
\ifx \advisorCnameC  \itsempty
\relax % 沒有 C 教授,所以不佔版面,不印任何空白
\else
% 共同指導的教授 C
\makebox[3cm][s]{}
\makebox[3cm][l]{\Large{\advisorCnameC}}  %顯示指導教授B中文名
\hspace{0.5cm}
\makebox[3cm][s]{}
\makebox[4cm][l]{\Large{\advisorEnameC}}
\hfill
\fi
\end{flushleft}
%
\vspace{1cm}
\Large{\univCname}\\
\vspace{0.3cm}
\Large{\deptCname}\\
\vspace{0.3cm}
\Large{\degreeCname 論文}\\
\vspace{2cm}
%
\normalsize{
A Thesis\\
\vspace{0.2cm}
Submitted to \fulldeptEname \\
\vspace{0.2cm}
\collEname \\
\vspace{0.2cm}
\univEname \\
\vspace{0.2cm}
in partial Fulfillment of the Requirements \\
\vspace{0.2cm}
for the Degree of \\
\vspace{0.2cm}
\degreeEname \\
\vspace{0.2cm}
in\\
\vspace{0.5cm}
\deptEname\\
\vspace{0.5cm}
\eMonth \  \eYear\\
\vspace{0.5cm}
\ePlace \\
}
\vspace{2cm}
%
\vfill
\makebox[10cm][s]{\Large{中華民國\cYear 年\cMonth 月}}%
%
\end{center}
% Resume the line spacing to the desired setting
\renewcommand{\baselinestretch}{\mybaselinestretch}   %恢復原設定
% it needs a font size changing command to be effective
% restore the font size to normal
\normalsize
\end{titlepage}

%%%%%%%%%%%%%%%%%%%%%%%%%%%%%%%
%       指導教授推薦書 
%%%%%%%%%%%%%%%%%%%%%%%%%%%%%%%
%
% insert the printed standard form when the thesis is ready to bind
% 在口試完成後,再將已簽名的推薦書放入以便裝訂
% create an entry in table of contents for 推薦書
% 目前送出空白頁
%\newpage{\thispagestyle{empty}\addcontentsline{toc}{chapter}{\nameInnerCover}\mbox{}\clearpage}
%\newpage

% 判斷是否要浮水印?
%\ifx\mywatermark\undefined 
%  \thispagestyle{empty}  % 無頁碼、無 header (無浮水印)
%\else
%  \thispagestyle{EmptyWaterMarkPage} % 無頁碼、有浮水印
%\fi

%%%%%%%%%%%%%%%%%%%%%%%%%%%%%%%%%%%%%%%%%%%%%%%%%%%%%%%%%%%%%%%
%%no page number
%% create an entry in table of contents for 書名頁
%\addcontentsline{toc}{chapter}{\nameInnerCover}
%
%
%% aligned to the center of the page
%\begin{center}
%% font size (relative to 12 pt):
%% \large (14pt) < \Large (18pt) < \LARGE (20pt) < \huge (24pt)< \Huge (24 pt)
%% Set the line spacing to single for the titles (to compress the lines)
%\renewcommand{\baselinestretch}{1}   %行距 1 倍
%% it needs a font size changing command to be effective
%%中文題目
%\Large{\cTitle}\\ %%%%%
%\vspace{1cm}
%% 英文題目
%\Large{\eTitle}\\ %%%%%
%%\vspace{1cm}
%\vfill
%% \makebox is a text box with specified width;
%% option s: stretch
%% use \makebox to make sure
%% 「研究生:」 與「指導教授:」occupy the same width
%\makebox[3cm][s]{\large{研 究 生:}}
%\makebox[3cm][l]{\large{\myCname}} %%%%%
%\hfill
%\makebox[2cm][s]{\large{Student: }}
%\makebox[5cm][l]{\large{\myEname}}\\ %%%%%
%%
%%\vspace{1cm}
%%
%\makebox[3cm][s]{\large{指導教授:}}
%\makebox[3cm][l]{\large{\advisorCnameA}} %%%%%
%\hfill
%\makebox[2cm][s]{\large{Advisor: }}
%\makebox[5cm][l]{\large{\advisorEnameA}}\\ %%%%%
%%
%% 判斷是否有共同指導的教授 B
%\ifx \advisorCnameB  \itsempty
%\relax % 沒有 B 教授,所以不佔版面,不印任何空白
%\else
%%共同指導的教授B
%\makebox[3cm][s]{}
%\makebox[3cm][l]{\large{\advisorCnameB}} %%%%%
%\hfill
%\makebox[2cm][s]{}
%\makebox[5cm][l]{\large{\advisorEnameB}}\\ %%%%%
%\fi
%%
%% 判斷是否有共同指導的教授 C
%\ifx \advisorCnameC  \itsempty
%\relax % 沒有 C 教授,所以不佔版面,不印任何空白
%\else
%%共同指導的教授C
%\makebox[3cm][s]{}
%\makebox[3cm][l]{\large{\advisorCnameC}} %%%%%
%\hfill
%\makebox[2cm][s]{}
%\makebox[5cm][l]{\large{\advisorEnameC}}\\ %%%%%
%\fi
%%
%% Resume the line spacing to the desired setting
%\renewcommand{\baselinestretch}{\mybaselinestretch}   %恢復原設定
%\large %it needs a font size changing command to be effective
%%
%\vfill
%\makebox[4cm][s]{\large{\univCname}}\\% 校名
%\makebox[6cm][s]{\large{\deptCname}}\\% 系所名
%\makebox[3cm][s]{\large{\degreeCname 論文}}\\% 學位名
%%
%%\vspace{1cm}
%\vfill
%\large{A Thesis}\\%
%\large{Submitted to }%
%%
%\large{\fulldeptEname}\\%系所全名 (英文)
%%
%%
%\ifx \collEname  \itsempty
%\relax % 沒有學院名 (英文),所以不佔版面,不印任何空白
%\else
%% 有學院名 (英文)
%\large{\collEname}\\% 學院名 (英文)
%\fi
%%
%\large{\univEname}\\%校名 (英文)
%%
%\large{in Partial Fulfillment of the Requirements}\\
%%
%\large{for the Degree of}\\
%%
%\large{\degreeEname}\\%學位名(英文)
%
%\large{in}\\
%%
%\large{\deptEname}\\%系所短名(英文;表明學位領域)
%%
%\large{\eMonth\ \eYear}\\%月、年 (英文)
%%
%\large{\ePlace}% 學校所在地 (英文)
%\vfill
%\large{中華民國}%
%\large{\cYear}% %%%%%
%\large{年}%
%\large{\cMonth}% %%%%%
%\large{月}\\
%\end{center}
%% restore the font size to normal
%\normalsize
%\clearpage


%%%%%%%%%%%%%%%%%%%%%%%%%%%%%%%%%%%%%%%%%%%%%%%%%%%%%%%%%%%%%%%%%%%%%
%%%%%%%%%%%%%%%%%%%%%%%%%%%%%%%
%       論文口試委員審定書 (計頁碼,但不印頁碼) 
%%%%%%%%%%%%%%%%%%%%%%%%%%%%%%%
%
% insert the printed standard form when the thesis is ready to bind
% 在口試完成後,再將已簽名的審定書放入以便裝訂
% create an entry in table of contents for 審定書
% 目前送出空白頁
%\newpage{\thispagestyle{empty}\addcontentsline{toc}{chapter}{\nameCommitteeForm}\mbox{}\clearpage}


%%%%%%%%%%%%%%
%% 從摘要到本文之前的部份以小寫羅馬數字印頁碼
% 但是從「書名頁」(但不印頁碼) 就開始計算
\setcounter{page}{1}
\pagenumbering{Roman}

%%%%%%%%%%%%%%%%%%%%%%%%%%%%%%%
%       中文摘要 
%%%%%%%%%%%%%%%%%%%%%%%%%%%%%%%
%
\newpage
\thispagestyle{plain}  % 無 header,但在浮水印模式下會有浮水印
% create an entry in table of contents for 中文摘要
\addcontentsline{toc}{chapter}{\nameCabstract}

% aligned to the center of the page
\begin{center}
% font size (relative to 12 pt):
% \large (14pt) < \Large (18pt) < \LARGE (20pt) < \huge (24pt)< \Huge (24 pt)
% Set the line spacing to single for the names (to compress the lines)
\renewcommand{\baselinestretch}{1}   %行距 1 倍
% it needs a font size changing command to be effective
\Large{\cTitle}\\  %中文題目
\vspace{1cm}
% \makebox is a text box with specified width;
% option s: stretch
% use \makebox to make sure
% each text field occupies the same width
\makebox[1.5cm][s]{\large{學生:}}
\makebox[3cm][l]{\large{\myCname}} %學生中文姓名
\hfill
%
\makebox[3cm][s]{\large{指導教授:}}
\makebox[3cm][l]{\large{\advisorCnameA}} \\ %教授A中文姓名
%
% 判斷是否有共同指導的教授 B
\ifx \advisorCnameB  \itsempty
\relax % 沒有 B 教授,所以不佔版面,不印任何空白
\else
%共同指導的教授B
\makebox[1.5cm][s]{}
\makebox[3cm][l]{} %%%%%
\hfill
\makebox[3cm][s]{}
\makebox[3cm][l]{\large{\advisorCnameB}}\\ %教授B中文姓名
\fi
%
% 判斷是否有共同指導的教授 C
\ifx \advisorCnameC  \itsempty
\relax % 沒有 C 教授,所以不佔版面,不印任何空白
\else
%共同指導的教授C
\makebox[1.5cm][s]{}
\makebox[3cm][l]{} %%%%%
\hfill
\makebox[3cm][s]{}
\makebox[3cm][l]{\large{\advisorCnameC}}\\ %教授C中文姓名
\fi
%
\vspace{1cm}
%
\large{\univCname}\large{\deptCname}
\large{﹙研究所﹚}\large{\degreeCname 班}\\ %校名系所名
\vspace{1cm}
%\vfill
\makebox[2.5cm][s]{\Large{摘要}}\\
\end{center}
% Resume the line spacing to the desired setting
\renewcommand{\baselinestretch}{\mybaselinestretch}   %恢復原設定
%it needs a font size changing command to be effective
% restore the font size to normal
\normalsize
%%%%%%%%%%%%%
這邊是中文摘要

%%%%%%%%%%%%%%%%%%%%%%%%%%%%%%%
%       英文摘要 
%%%%%%%%%%%%%%%%%%%%%%%%%%%%%%%
%
\newpage
\thispagestyle{plain}  % 無 header,但在浮水印模式下會有浮水印

% create an entry in table of contents for 英文摘要
\addcontentsline{toc}{chapter}{\nameEabstract}

% aligned to the center of the page
\begin{center}
% font size:
% \large (14pt) < \Large (18pt) < \LARGE (20pt) < \huge (24pt)< \Huge (24 pt)
% Set the line spacing to single for the names (to compress the lines)
\renewcommand{\baselinestretch}{1}   %行距 1 倍
%\large % it needs a font size changing command to be effective
\Large{\eTitle}\\  %英文題目
\vspace{1cm}
% \makebox is a text box with specified width;
% option s: stretch
% use \makebox to make sure
% each text field occupies the same width
\makebox[2cm][s]{\large{Student: }}
\makebox[5cm][l]{\large{\myEname}} %學生英文姓名
\hfill
%
\makebox[2cm][s]{\large{Advisor: }}
\makebox[5cm][l]{\large{\advisorEnameA}} \\ %教授A英文姓名
%
% 判斷是否有共同指導的教授 B
\ifx \advisorCnameB  \itsempty
\relax % 沒有 B 教授,所以不佔版面,不印任何空白
\else
%共同指導的教授B
\makebox[2cm][s]{}
\makebox[5cm][l]{} %%%%%
\hfill
\makebox[2cm][s]{}
\makebox[5cm][l]{\large{\advisorEnameB}}\\ %教授B英文姓名
\fi
%
% 判斷是否有共同指導的教授 C
\ifx \advisorCnameC  \itsempty
\relax % 沒有 C 教授,所以不佔版面,不印任何空白
\else
%共同指導的教授C
\makebox[2cm][s]{}
\makebox[5cm][l]{} %%%%%
\hfill
\makebox[2cm][s]{}
\makebox[5cm][l]{\large{\advisorEnameC}}\\ %教授C英文姓名
\fi
%
\vspace{1cm}
\large{Department﹙Institute﹚of }\large{\deptEname}\\  %英文系所全名
%
%\ifx \collEname  \itsempty
%\relax % 如果沒有學院名 (英文),則不佔版面,不印任何空白
%\else
% 有學院名 (英文)
%\large{\collEname}\\% 學院名 (英文)
%\fi
%
\large{\univEname}\\  %英文校名
\vspace{1cm}
%\vfill
%
\Large{ABSTRACT}\\
%\vspace{0.5cm}
\end{center}
% Resume the line spacing the desired setting
\renewcommand{\baselinestretch}{\mybaselinestretch}   %恢復原設定
%\large %it needs a font size changing command to be effective
% restore the font size to normal
\normalsize
%%%%%%%%%%%%%
This is English abstract.

%%%%%%%%%%%%%%%%%%%%%%%%%%%%%%%
%       誌謝 
%%%%%%%%%%%%%%%%%%%%%%%%%%%%%%%
%
% Acknowledgment
\newpage
\chapter*{\protect\makebox[5cm][s]{\nameAckn}} %\makebox{} is fragile; need protect
\addcontentsline{toc}{chapter}{\nameAckn}
謝天, 謝地, 乾蝦大家. 也可以謝謝女朋友, 在就學期間沒有出來干擾我, 讓我可以順利完成學業. (咦?)

%%%%%%%%%%%%%%%%%%%%%%%%%%%%%%%
%       目錄 
%%%%%%%%%%%%%%%%%%%%%%%%%%%%%%%
%
% Table of contents
\newpage
\renewcommand{\contentsname}{\protect\makebox[5cm][s]{\nameToc}}
%\makebox{} is fragile; need protect
\addcontentsline{toc}{chapter}{\nameToc}
\tableofcontents

%%%%%%%%%%%%%%%%%%%%%%%%%%%%%%%
%       表目錄 
%%%%%%%%%%%%%%%%%%%%%%%%%%%%%%%
%
% List of Tables
\newpage
\renewcommand{\listtablename}{\protect\makebox[5cm][s]{\nameLot}}
%\makebox{} is fragile; need protect
\addcontentsline{toc}{chapter}{\nameLot}
\listoftables

%%%%%%%%%%%%%%%%%%%%%%%%%%%%%%%
%       圖目錄 
%%%%%%%%%%%%%%%%%%%%%%%%%%%%%%%
%
% List of Figures
\newpage
\renewcommand{\listfigurename}{\protect\makebox[5cm][s]{\nameTof}}
%\makebox{} is fragile; need protect
\addcontentsline{toc}{chapter}{\nameTof}
\listoffigures

%%%%%%%%%%%%%%%%%%%%%%%%%%%%%%%
%       演算法目錄 
%%%%%%%%%%%%%%%%%%%%%%%%%%%%%%%
%
% List of Figures
\newpage
\renewcommand{\listalgorithmname}{\protect\makebox[5cm][s]{\nameToa}}
%\makebox{} is fragile; need protect
\addcontentsline{toc}{chapter}{\nameToa}
\listofalgorithms


%%%%%%%%%%%%%%%%%%%%%%%%%%%%%%%
%       符號說明 
%%%%%%%%%%%%%%%%%%%%%%%%%%%%%%%
%
% Symbol list
% define new environment, based on standard description environment
% adapted from p.60~64, <<The LaTeX Companion>>, 1994, ISBN 0-201-54199-8
%\newcommand{\SymEntryLabel}[1]%
% {\makebox[3cm][l]{#1}}
%
%\newenvironment{SymEntry}
%   {\begin{list}{}%
%       {\renewcommand{\makelabel}{\SymEntryLabel}%
%        \setlength{\labelwidth}{3cm}%
%        \setlength{\leftmargin}{\labelwidth}%
%        }%
%   }%
%   {\end{list}}
%%
%\newpage
%\chapter*{\protect\makebox[5cm][s]{\nameSlist}} %\makebox{} is fragile; need protect
%\addcontentsline{toc}{chapter}{\nameSlist}
%%
% this file is encoded in utf-8
% v1.7
%  各符號以 \item[] 包住,然後接著寫說明
% 如果符號是數學符號,應以數學模式表示,以取得正確的字體
% 如果符號本身帶有方括號,則此符號可以用大括號 {} 包住保護
\begin{SymEntry}

\item[OLED]
Organic Light Emitting Diode

\item[$E$]
energy

\item[$e$]
the absolute value of the electron charge, $1.60\times10^{-19}\,\text{C}$
 
\item[$\mathscr{E}$]
electric field strength (V/cm)

\item[{$A[i,j]$}]
the  element of the matrix $A$ at $i$-th row, $j$-th column\\
矩陣 $A$ 的第 $i$ 列,第 $j$ 行的元素

\end{SymEntry}


\newpage
%% 論文本體頁碼回復為阿拉伯數字計頁,並從頭起算
\pagenumbering{arabic}
%%%%%%%%%%%%%%%%%%%%%%%%%%%%%%%%